%%	This is file 'fancyhandout-doc.tex', Version 2018-01-18
%%	Copyright 2017-18 Sebastian Friedl <sfr682k@t-online.de>
%% 
%%	This work may be distributed and/or modified under the conditions of the LaTeX Project
%%	Public License, either version 1.3c of this license or (at your option) any later version.
%%	The latest version of this license is available at
%%		http://www.latex-project.org/lppl.txt
%%	and version 1.3c or later is part of all distributions of LaTeX version 2008-05-04 or later.
%%
%%	This work has the LPPL maintenance status 'maintained'.
%%	Author: Sebastian Friedl
%%	Current maintainer of this work is Sebastian Friedl
%%
%%	This work consists of the files fancyhandout.cls and fancyhandout-doc.tex
%%
%%	---------------------------------------------------------------------------------------------------------------------------------------------
%%
%%	This class and its documentation are currently under construction.
%%	Commands, options and behaviour may change with future versions of the class.
%%
%%	---------------------------------------------------------------------------------------------------------------------------------------------
%%
%%	Please report bugs and other problems as well as suggestions for improvements to the following email address: sfr682k@t-online.de
%%
%%	--------------------------------------------------------------------------------------------------------------------------------------------- 


\documentclass[11pt]{ltxdoc}

\usepackage{csquotes}
\usepackage[unicode, pdfborder={0 0 0}]{hyperref}
\usepackage{verbatim}

\usepackage[erewhon]{newtxmath}
\usepackage{fontspec}
\setmainfont{erewhon}
\setsansfont[Scale=MatchLowercase]{Source Sans Pro}
\setmonofont[Scale=MatchLowercase]{Hack}

\usepackage{polyglossia}
\setdefaultlanguage{english}
\usepackage[english]{selnolig}


\MakeShortVerb{"}
\parindent0pt

\usepackage[left=4.50cm,right=2.75cm,top=3.25cm,bottom=2.75cm,nohead]{geometry}

\hyphenation{re-du-cing}

\title{The \texttt{fancyhandout} class \\ {\large\url{https://github.com/SFr682k/fancyhandout}}}
\author{Sebastian Friedl \\ \href{mailto:sfr682k@t-online.de}{\ttfamily sfr682k@t-online.de}}
\date{2018/01/18}

\hypersetup{pdftitle={The fancyhandout class},pdfauthor={Sebastian Friedl}}

\begin{document}
	\maketitle
	
	\begin{abstract}
		\noindent%
		A \LaTeX\ class for typesetting fancy handouts.
	\end{abstract}
	
	This class and its documentation are currently under construction. \\
	Commands, options and behaviour may change with future versions of the class.
	
	\tableofcontents
	\clearpage
	
	
	
	\subsection*{Dependencies on other packages}
	\addcontentsline{toc}{subsection}{Dependencies on other packages}
	The "fancyhandout" class requires \LaTeXe\ and the following packages:
	\begin{multicols}{3}\ttfamily\centering
		enumitem \\ fancyhdr \\ geometry \\ xcolor
	\end{multicols}
	
	\subsection*{License}
	\addcontentsline{toc}{subsection}{License}
	\textcopyright\ 2017-18 Sebastian Friedl
	
	\smallskip
	This work may be distributed and/or modified under the conditions of the \LaTeX\ Project Public License, either version 1.3c of this license or (at your option) any later version.
	
	\smallskip
	The latest version of this license is available at \url{http://www.latex-project.org/lppl.txt} and version 1.3c or later is part of all distributions of \LaTeX\ version 2008-05-04 or later.
	
	\smallskip
	This work has the LPPL maintenance status \enquote*{maintained}. The current maintainer of this work is Sebastian Friedl. \\
	This work consists of the following files:
	\begin{itemize} \itemsep 0pt
		\item "fancyhandout.cls" and
		\item "fancyhandout-doc.tex"
	\end{itemize}


	\subsection*{Call for cooperation}
	\addcontentsline{toc}{subsection}{Call for cooperation}
	Please report bugs and other problems as well as suggestions for improvements by using the \href{https://github.com/SFr682k/fancyhandout/issues}{issue tracker on GitHub} or sending an email to \href{mailto:sfr682k@t-online.de}{\texttt{sfr682k@t-online.de}}.


	\clearpage
	
	
	
	% DOCUMENTATION PART ----------------------------------------------------------------------
	
	\section{Loading \texttt{fancyhandout}}
	Load "fancyhandout" by using the "\documentclass" command.
	
	\bigskip
	By default, "fancyhandout" typesets documents using two side page layout on DIN/ISO A4 paper and an 11pt \textsf{sans serif} font. This can be changed by using these class options:
	
	\medskip
	\DescribeMacro{10pt}\DescribeMacro{12pt}
	These options change the font size to 10 or 12 points. \\
	Please do \emph{not} use both options simultaneously.
	
	\medskip
	\DescribeMacro{rmfont}
	This option changes the used font to the roman (=\,serif) one.
	
	\medskip
	\DescribeMacro{letter}
	This option changes the page size to letter format.
	
	\medskip
	\DescribeMacro{oneside}
	This option sets leads to using one side page layout.
	
	
	
	\section{Other configuration possibilities}
	\begin{itemize}
		\item \textbf{Lists} \\
			By default, "fancyhandout" loads "enumitem" for setting "itemsep" to~"0ex" and reducing "topsep" to~".75ex". \\
			These settings can be reverted or modified using the "\setlist" command (see the "itemsep" package documentation for further details).
		
		\item \textbf{Page margins} \\
			"fancyhandout" uses the "geometry" package for setting the page margins to 2.25\,cm each, including page head and foot. \\
			These margins (and the page geometry) can be modified using the "\geometry" command (see the "geometry" package documentation for further details).
			
			\smallskip
			Also, it is possible to construct the page area by loading the "typearea" package.
		
		\item \textbf{Paragraph indent} \\
			The length of "\parindent" is set to "0pt" when loading "fancyhandout". If you want to change it, simply use some command like "\parindent 1em" after the class has been loaded to set "\parindent"'s length to it's initial value.
	\end{itemize}
	
	
	\section{Creating a handout}
	"fancyhandout" currently behaves just like the \enquote{normal} article class. \\
	However, there is no possibility to pass short section, subsection and subsubsection names in square brackets.
	
	\bigskip
	As soon as there are more relevant changes, this section will contain a step-by-step introduction on using "fancyhandout".
	
	
	\clearpage
	\section{Upcoming features}
	Following features are planned to be finished with the next versions of "fancyhandout":
	\begin{enumerate}
		\item
			Define an "\institute" command.
		
		\item
			Make used colors changeable
		
		\item
			Improve the appereance of "\maketitle"?
		
		\item
			Provide "beamer"-like boxes
	\end{enumerate}
	
	The current aim is to release an initial version with basic functionality on CTAN before Feburary 28, 2018.
\end{document}

