%%	This is file 'fancyhandout-doc.tex', Version 2018-01-22
%%	Copyright 2017-18 Sebastian Friedl <sfr682k@t-online.de>
%% 
%%	This work may be distributed and/or modified under the conditions of the LaTeX Project
%%	Public License, either version 1.3c of this license or (at your option) any later version.
%%	The latest version of this license is available at
%%		http://www.latex-project.org/lppl.txt
%%	and version 1.3c or later is part of all distributions of LaTeX version 2008-05-04 or later.
%%
%%	This work has the LPPL maintenance status 'maintained'.
%%	Author: Sebastian Friedl
%%	Current maintainer of this work is Sebastian Friedl
%%
%%	This work consists of the files fancyhandout.cls and fancyhandout-doc.tex
%%
%%	---------------------------------------------------------------------------------------------------------------------------------------------
%%
%%	A LaTeX class for producing nice-looking handouts.
%%
%%	---------------------------------------------------------------------------------------------------------------------------------------------
%%
%%	Please report bugs and other problems as well as suggestions for improvements to the following email address: sfr682k@t-online.de
%%
%%	--------------------------------------------------------------------------------------------------------------------------------------------- 


\documentclass[11pt]{ltxdoc}

\usepackage{csquotes}
\usepackage[unicode, pdfborder={0 0 0}]{hyperref}
\usepackage{verbatim}

\usepackage[erewhon]{newtxmath}
\usepackage{fontspec}
\setmainfont{erewhon}
\setsansfont[Scale=MatchLowercase]{Source Sans Pro}
\setmonofont[Scale=MatchLowercase]{Hack}

\usepackage{polyglossia}
\setdefaultlanguage{english}
\usepackage[english]{selnolig}


\MakeShortVerb{"}
\parindent0pt

\usepackage[left=4.50cm,right=2.75cm,top=3.25cm,bottom=2.75cm,nohead]{geometry}

\hyphenation{re-du-cing}

\title{The \texttt{fancyhandout} class \\ {\large\url{https://github.com/SFr682k/fancyhandout}}}
\author{Sebastian Friedl \\ \href{mailto:sfr682k@t-online.de}{\ttfamily sfr682k@t-online.de}}
\date{2018/01/22}

\hypersetup{pdftitle={The fancyhandout class},pdfauthor={Sebastian Friedl}}

\begin{document}
	\maketitle
	
	\begin{abstract}\parindent0pt%
		\noindent%
		\textbf{A \LaTeX\ class for producing nice-looking handouts.}
		
		\smallskip
		Many students conducting a presentation are asked to provide a handout containing the essential information of their report.
		However, \LaTeX\ has never been designed for typesetting such documents mostly consisting out of bullet points.
		
		\smallskip
		"fancyhandout" breaks up with some of \LaTeX's principles and redefines basic \LaTeX\ commands with the aim of producing well-designed and clear structured handouts: \\
		A \textsf{sans-serif} font is used by default, sections are not numbered and highlighted by underlining them, head- and footline display document information and for avoiding too large whitespace around the text, the margin sizes are adjusted to a proper value.
		
		\smallskip
		All together, "fancyhandout" provides a way of typesetting documents not exclusively consisting of running text in a beautiful way.
	\end{abstract}

	
	\tableofcontents
	\clearpage
	
	
	
	\subsection*{Dependencies on other packages}
	\addcontentsline{toc}{subsection}{Dependencies on other packages}
	The "fancyhandout" class requires \LaTeXe\ and the following packages:
	\begin{multicols}{3}\ttfamily\centering
		csquotes \\ enumitem \\ etoolbox \\ fancyhdr \\ geometry \\ xcolor
	\end{multicols}
	
	\subsection*{License}
	\addcontentsline{toc}{subsection}{License}
	\textcopyright\ 2017-18 Sebastian Friedl
	
	\smallskip
	This work may be distributed and/or modified under the conditions of the \LaTeX\ Project Public License, either version 1.3c of this license or (at your option) any later version.
	
	\smallskip
	The latest version of this license is available at \url{http://www.latex-project.org/lppl.txt} and version 1.3c or later is part of all distributions of \LaTeX\ version 2008-05-04 or later.
	
	\smallskip
	This work has the LPPL maintenance status \enquote*{maintained}. The current maintainer of this work is Sebastian Friedl. \\
	This work consists of the following files:
	\begin{itemize} \itemsep 0pt
		\item "fancyhandout.cls" and
		\item "fancyhandout-doc.tex"
	\end{itemize}


	\subsection*{Call for cooperation}
	\addcontentsline{toc}{subsection}{Call for cooperation}
	Please report bugs and other problems as well as suggestions for improvements by using the \href{https://github.com/SFr682k/fancyhandout/issues}{issue tracker on GitHub} or sending an email to \href{mailto:sfr682k@t-online.de}{\texttt{sfr682k@t-online.de}}.


	\clearpage
	
	
	
	% DOCUMENTATION PART ----------------------------------------------------------------------
	
	\section{Creating a handout}
	\subsection{Loading \texttt{fancyhandout}}
	Load "fancyhandout" by using the "\documentclass" command.
	
	\bigskip
	By default, "fancyhandout" typesets documents using two side page layout on DIN/ISO A4 paper and an 11pt \textsf{sans serif} font. This can be changed by using these class options:
	
	\medskip
	\DescribeMacro{10pt}\DescribeMacro{12pt}
	These options change the font size to 10 or 12 points. \\
	Please do \emph{not} use both options simultaneously.
	
	\medskip
	\DescribeMacro{rmfont}
	This option changes the used font to the roman (=\,serif) one.
	
	\medskip
	\DescribeMacro{letter}
	This option changes the page size to letter format.
	
	\medskip
	\DescribeMacro{oneside}
	This option leads to using one side page layout.
	
	
	\subsection{Providing document information}\label{docinfo}
	Basic \LaTeX\ defines the commands "\title", "\author" and "\date" for providing document information.
	When using "fancyhandout" the following ones are available:
	
	\medskip
	\DescribeMacro{\title}
	Sets the handout's title. \\[\smallskipamount]
	\textit{Example:}\quad "\title{An introduction on \texttt{fancyhandout}}"
	
	\medskip
	\DescribeMacro{\subtitle}
	When required, a subtitle may be also provided. \\[\smallskipamount]
	\textit{Example:}\quad "\subtitle{With some notes on \LaTeX\ itself}"
	
	\medskip
	\DescribeMacro{\author}
	Useful for specifying the author(s). \\
	Multiple authors should be separated using the "\and" command inside "\author" (e.~g. "\author{J. Doe \and M. Mustermann}"). \\
	"\and" will expand to "\qquad" in the title and to ",~" in the headline. \\[\smallskipamount]
	\textit{Example:}\quad "\author{Sebastian Friedl}"
	
	\medskip
	\DescribeMacro{\institute}
	Additional to or instead of the author(s), it is also possible to specify the institute name(s). \\
	You may also use the "\and" command to separate different names. \\
	However, information about the institute won't be shown in the headline. \\[\smallskipamount]
	\textit{Example:}\quad "\institute{University of Foo Bar City}"
	
	\medskip
	\DescribeMacro{\date}
	This command changes the date shown in the title and footline.
	"\date"'s value is "\today" until another value for "\date" is given. \\[\smallskipamount]
	\textit{Example:}\quad "\date{20. January 2018}"
	
	\subsubsection*{Short version}
	It is also possible to provide short versions of the command's values as an optional argument in squared braces.
	These are only used inside the head- or footline. \\[\smallskipamount]
	\textit{Example:}\quad "\author[S. Friedl]{Sebastian Friedl}"
	
	
	\subsection{The handout title}
	\DescribeMacro{\maketitle}
	Using the "\maketitle" command, a colored title box displaying the document information listed in section \ref{docinfo} gets typeset.
	
	\medskip
	The colors of the box can be changed by "\colorlet"ting …
	\begin{itemize}
		\item "fancyhandouttboxlinecolor" \\
			to change the color used for drawing the line around the title box \\[\smallskipamount]
			\textit{Example:}\quad "\colorlet{fancyhandouttboxlinecolor}{red!80!black}"
		
		\item "fancyhandouttboxfillcolor" \\
			to change the color used for filling the title box \\[\smallskipamount]
			\textit{Example:}\quad "\colorlet{fancyhandouttboxfillcolor}{red!10}"
	\end{itemize}
	
	
	\subsection{Sectioning}
	\DescribeMacro{\section}\DescribeMacro{\subsection}\DescribeMacro{\subsubsection}
	"fancyhandout" supports the three basic sectioning commands "\section", "\subsection" and "\subsubsection". These commands produce sections, subsections and subsubsections visible in the table of contents (TOC). \\
	\DescribeMacro{\section*}\DescribeMacro{\subsection*}\DescribeMacro{\subsubsection*}
	The starred versions "\section*", "\subsection*" and "\subsubsection*" are also provided to produce sections, subsections and subsubsections not visible in the table of contents.
	
	\medskip
	All sections, subsections and subsubsections are \emph{not} numbered. \\
	Section and subsection headings are underlined to stand out. The colors of these lines can be changed by "\colorlet"ting …
	\begin{itemize}
		\item "fancyhandoutsectlinecolor" \\
			to change the color used for underlining section headings \\[\smallskipamount]
			\textit{Example:}\quad "\colorlet{fancyhandoutsectlinecolor}{red!80!black}"
		
		\item "fancyhandoutsubsectlinecolor" \\
			to change the color used for underlining subsection headings \\[\smallskipamount]
			\textit{Example:}\quad "\colorlet{fancyhandoutsubsectlinecolor}{red!70!black!70}"
	\end{itemize}
	
	\medskip
	\textbf{Important note:} \\
	Unlike "article"'s sectioning commands, "fancyhandout"'s sectioning commands do \emph{not} accept optional arguments in squared braces.
	
	
	\subsection{The main content}
	The main content may be any arbitrary \LaTeX\ code.
	
	\medskip
	\textbf{Note:} \\
	Neither the sectioning commands, nor the table of contents print section, subsection and subsubsection numbers in front of section headings. \\
	Therefore, relying on section numbers when numbering things like figures and equations is usually a very, \emph{very} bad idea. Rely on page numbers instead.
	
	
	
	
	\section{Advanced settings}
	\subsection{Lists}
	By default, "fancyhandout" loads "enumitem" for setting "itemsep" to~"0ex" and reducing "topsep" to~".75ex". \\
	\DescribeMacro{\setlist}
	These settings can be reverted or modified using the "\setlist" command (see "enumitem"'s package documentation for further details).
	
	\subsection{Page margins}
	"fancyhandout" uses the "geometry" package for setting the page margins to 2.25\,cm each, including page head and foot. \\
	\DescribeMacro{\geometry}
	These margins (and the page geometry) can be modified using the "\geometry" command (see "geometry"'s package documentation for further details).
	
	\smallskip
	\DescribeMacro{typearea}
	Also, it is possible to construct the page area by loading the "typearea" package.
	
	\subsection{Paragraph indent}
	\DescribeMacro{\parindent}
	The length of "\parindent" is set to "0pt" when loading "fancyhandout". If you want to change it, simply use a command like "\parindent1em" after the class has been loaded to set "\parindent"'s length to it's initial value.
	
	
	
	
	\clearpage
	\section{Upcoming features}
	The following features are planned to be finished with the next versions of "fancyhandout":
	\begin{enumerate}
		\item
			Provide different styles for "\maketitle" \\
			\textit{e.~g. centered text above a separation line}
		
		\item
			Provide "beamer"-like block environments
		
		\item
			Provide color themes \\
			\textit{e.~g. red, green, …}
	\end{enumerate}
	
	You may add other feature requests of public interest to the \href{https://github.com/SFr682k/fancyhandout/issues}{issue tracker on GitHub}.
\end{document}

